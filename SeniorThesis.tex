% RUN latexmk -lualatex
% もっとLuaLaTeXを使いなさいの考えからLuaLaTeX + BibLaTeXでフォーマットを合わせるべくそれなりの無茶をしている。
% 特に拘りがないならあぴす(@apissQX70)氏のテンプレートをちょっとカスタムするのが楽かと。
% Thesis style REF: https://github.com/apiss2/master_thesis_tmplete

\documentclass[a4paper, book, twoside, openany, fontsize = 11pt, gutter = 3cm]{jlreq}
	% Settings for jlreq style
		\jlreqsetup
		{
			appendix_counter = {
				section = {value = 0}
			},
		}
		\ModifyPageStyle{plain} % jlreq setting
		{
			nombre_position = top-right % ノンブル位置 右肩=右上?
		}

	% 基礎科学実験Aの置物 Memo
		% 1: \rmは非推奨。\mathrmが望ましい。
		% 2: center環境は非推奨。\centeringが望ましい。
		% 3: eqnarray環境はamsmathが対応していない。amsmathはalign環境を使用。

	% Packages
		\usepackage{url} % URL typeset
		\usepackage{amsmath, amssymb}
		\usepackage{mathtools}
		\usepackage{here} % Force position with [H]
		\usepackage{graphicx} % for pictures. LuaLaTex don't use dvipdfmx.
		\usepackage{bm} % short \boldsymbol{}
		\usepackage[subrefformat=parens]{subcaption}
		\usepackage{enumitem}
		\usepackage[backend=biber, sorting=none, bibencoding=utf8, style=ieee, doi=false, isbn=false]{biblatex}
		% WARNING: bliography use bibLAtex NOT bibtex
		\usepackage[framed, numbered, autolinebreaks]{mcode} % for MATLAB Source
		% mcode REF:https://ssr-yuki.hatenablog.com/entry/2019/04/15/162517
		\usepackage[T1]{fontenc} % Latin Modern
		\usepackage{lmodern} % Latin Modern
		\usepackage[pdfencoding=auto, bookmarksnumbered=true, hidelinks]{hyperref} % Links on PDF

	% Def Commands
		\newcommand{\eeqref}{式~\eqref}
		\newcommand{\tbref}{表~\ref}
		\newcommand{\figref}{図~\ref}
		\newcommand{\lisref}{ソースコード~\ref}
		\newcommand{\secref}[1]{\ref{#1}節}
		\newcommand{\chapref}[1]{第\ref{#1}章}
		\renewcommand{\lstlistingname}{ソースコード}
		\DeclareMathOperator{\sinc}{sinc} % needs amsmath
	
	% BibLatex Settings
		% \renewbibmacro{in:}{} % In: Journal... to . Journal...
		\DeclareFieldFormat*{volume}{#1} % renove vol.
		\DeclareFieldFormat[article]{number}{#1} % remove no.
		\DeclareFieldFormat{journaltitle}{\textit{#1}} % Cancel Japanese jurnaltitle Bold
		\DeclareFieldFormat{title}{#1} % Cancel Japanese title Bold
		\renewbibmacro*{volume+number+eid} % set vol(no)
		{
			\textbf{\printfield{volume}} % no punctuation between
			\printtext[parens]{\printfield{number}} % volume and number fields
			\setunit{\bibeidpunct}
			\printfield{eid}
		}
		% \renewbibmacro*{title-in-maintitle} % Cancel book title Bold
		% {
  		% 	\ifboolexpr
		% 	{
   		% 		test {\iffieldundef{title}}
    	% 		and
    	% 		test {\iffieldundef{subtitle}}
  		% 	}
   		% 	{}
    	% 	{
      	% 		\printtext[title]
		% 		{%
    	% 			\printfield{title}%
        % 			\setunit{\subtitlepunct}%
        % 			\printfield{subtitle}%
      	% 		}%
      	% 		\newunit
    	% 	}%
		% }
		% \renewbibmacro*{date} % yyyy to (yyyy)
		% {
		% 	\iffieldundef{year}
		% 	{}
		% 	{
		% 		\printtext[parens]{\printfield{year}}
		% 	}
		% }
		
		\addbibresource{References.bib} % biblatex: bib file

	% Other Settings
		% \mathtoolsset{showonlyrefs=true} % refされない数式=番号なし
		\setcounter{tocdepth}{2} % include subsection to TOC

	% set title
		\title{
			令和n年度 卒業論文 \\
			\huge{タイトル\\が長い時\vspace{280pt}} % 題目 + 余白
		}
		\author{
			電気通信大学 情報理工学域
			% I類
			% I\hspace{-1.2pt}I類 % II類って書きたい
			I\hspace{-1.2pt}I\hspace{-1.2pt}I類 % III類って書きたい
			なんとかプログラム \\ 
			0000000 電通 花子 \\
			指導教員 電通 太郎 教授
		}
		\date{提出日 令和n年 1月 31日}

\begin{document}
	\frontmatter
		\maketitle
		\tableofcontents

	\mainmatter
		% MARK: Introduction
		\chapter{序論}
			\section{目的}
				本研究の目的は云々

			\section{本論文の構成}
				本論文は全\ref{chap:conclusion}章で構成されている。
				\chapref{chap:principle}で原理について述べる。
				\chapref{chap:experiment}で実験について述べる。
				\chapref{chap:conclusion}では本論文のまとめを述べる。

		% MARK: Principle
		\chapter{原理} \label{chap:principle}
			\section{原理1}
				これは\eeqref{eq:einstein}に従う。
				\begin{align}
					E = mc^2 \label{eq:einstein}
				\end{align}
				\subsection{原理1-1}
					これの原理はこう\cite{uec.ac.jp:sotsuron_example}です。
					\begin{align}
						\sinc x % sinc関数をぷりアンブルで定義してるのでこんなことができたり。
					\end{align}

			\section{原理2}
				これの原理はこうです。
				\begin{figure}[H]
					\centering
					\includegraphics[width=0.5\columnwidth]{image.png}
					\caption{イメージ}
					\label{fig:image}
				\end{figure}

		% MARK: Experiment
		\chapter{実験} \label{chap:experiment}
			\section{実験器具及び実験環境}
				\subsection{実験器具}
					\begin{itemize}
						\item 実験
						\item 器具
						\item 一覧
					\end{itemize}

			% MARK: Ex1
			\section{実験1}
				付録\ref{sec:matlabscript}に示す\lisref{matlabscript}を用いて実験した。
				結果を\tbref{tb:kekka}に示す。
				\begin{table}[H]
					\centering
					\caption{}
					\label{tb:kekka}
					\begin{tabular}{c|c}
						\hline
						A & B \\
						\hline
						$0.0$ & $0.0$ \\
						$0.0$ & $0.0$ \\
						\hline
					\end{tabular}
				\end{table}
				また\figref{fig:images}に示す。
				\begin{figure}[H]
					\centering
					\begin{minipage}[b]{0.45\columnwidth}
						\centering
						\includegraphics[width=0.96\columnwidth]{image.png}
						\subcaption{A}
						\label{fig:image1}
					\end{minipage}
					\begin{minipage}[b]{0.45\columnwidth}
						\centering
						\includegraphics[width=0.96\columnwidth]{image.png}
						\subcaption{B}
						\label{fig:image2}
					\end{minipage}
					\caption{イメージ}
					\label{fig:images}
				\end{figure}
				\figref{fig:image1}に対して、\figref{fig:image2}は云々。

		% MARK: Conclusion
		\chapter{結論} \label{chap:conclusion}
			という結論となった。

			今後の課題として、これが挙げられる。

	\backmatter
		\nocite{bib-example} % biblatex: 引用しない文献を表示

		\printbibliography[title=参考文献] % biblatex
		\chapter{謝辞}
			本研究を行うにあたって、
			1年間指導してくださった電通太郎教授に深く感謝いたします。
			また研究室の活動で大変お世話になった方々
			に心から感謝いたします。

		\appendix
		\chapter{付録}
			\section{MATLABスクリプト} \label{sec:matlabscript}
				\lstinputlisting[caption={MATLABスクリプト},label=matlabscript,captionpos=b]{matlabscript.m}

\end{document}